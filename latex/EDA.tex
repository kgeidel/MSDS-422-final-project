\documentclass[11pt]{article}

    \usepackage[breakable]{tcolorbox}
    \usepackage{parskip} % Stop auto-indenting (to mimic markdown behaviour)
    

    % Basic figure setup, for now with no caption control since it's done
    % automatically by Pandoc (which extracts ![](path) syntax from Markdown).
    \usepackage{graphicx}
    % Maintain compatibility with old templates. Remove in nbconvert 6.0
    \let\Oldincludegraphics\includegraphics
    % Ensure that by default, figures have no caption (until we provide a
    % proper Figure object with a Caption API and a way to capture that
    % in the conversion process - todo).
    \usepackage{caption}
    \DeclareCaptionFormat{nocaption}{}
    \captionsetup{format=nocaption,aboveskip=0pt,belowskip=0pt}

    \usepackage{float}
    \floatplacement{figure}{H} % forces figures to be placed at the correct location
    \usepackage{xcolor} % Allow colors to be defined
    \usepackage{enumerate} % Needed for markdown enumerations to work
    \usepackage{geometry} % Used to adjust the document margins
    \usepackage{amsmath} % Equations
    \usepackage{amssymb} % Equations
    \usepackage{textcomp} % defines textquotesingle
    % Hack from http://tex.stackexchange.com/a/47451/13684:
    \AtBeginDocument{%
        \def\PYZsq{\textquotesingle}% Upright quotes in Pygmentized code
    }
    \usepackage{upquote} % Upright quotes for verbatim code
    \usepackage{eurosym} % defines \euro

    \usepackage{iftex}
    \ifPDFTeX
        \usepackage[T1]{fontenc}
        \IfFileExists{alphabeta.sty}{
              \usepackage{alphabeta}
          }{
              \usepackage[mathletters]{ucs}
              \usepackage[utf8x]{inputenc}
          }
    \else
        \usepackage{fontspec}
        \usepackage{unicode-math}
    \fi

    \usepackage{fancyvrb} % verbatim replacement that allows latex
    \usepackage{grffile} % extends the file name processing of package graphics
                         % to support a larger range
    \makeatletter % fix for old versions of grffile with XeLaTeX
    \@ifpackagelater{grffile}{2019/11/01}
    {
      % Do nothing on new versions
    }
    {
      \def\Gread@@xetex#1{%
        \IfFileExists{"\Gin@base".bb}%
        {\Gread@eps{\Gin@base.bb}}%
        {\Gread@@xetex@aux#1}%
      }
    }
    \makeatother
    \usepackage[Export]{adjustbox} % Used to constrain images to a maximum size
    \adjustboxset{max size={0.9\linewidth}{0.9\paperheight}}

    % The hyperref package gives us a pdf with properly built
    % internal navigation ('pdf bookmarks' for the table of contents,
    % internal cross-reference links, web links for URLs, etc.)
    \usepackage{hyperref}
    % The default LaTeX title has an obnoxious amount of whitespace. By default,
    % titling removes some of it. It also provides customization options.
    \usepackage{titling}
    \usepackage{longtable} % longtable support required by pandoc >1.10
    \usepackage{booktabs}  % table support for pandoc > 1.12.2
    \usepackage{array}     % table support for pandoc >= 2.11.3
    \usepackage{calc}      % table minipage width calculation for pandoc >= 2.11.1
    \usepackage[inline]{enumitem} % IRkernel/repr support (it uses the enumerate* environment)
    \usepackage[normalem]{ulem} % ulem is needed to support strikethroughs (\sout)
                                % normalem makes italics be italics, not underlines
    \usepackage{mathrsfs}
    

    
    % Colors for the hyperref package
    \definecolor{urlcolor}{rgb}{0,.145,.698}
    \definecolor{linkcolor}{rgb}{.71,0.21,0.01}
    \definecolor{citecolor}{rgb}{.12,.54,.11}

    % ANSI colors
    \definecolor{ansi-black}{HTML}{3E424D}
    \definecolor{ansi-black-intense}{HTML}{282C36}
    \definecolor{ansi-red}{HTML}{E75C58}
    \definecolor{ansi-red-intense}{HTML}{B22B31}
    \definecolor{ansi-green}{HTML}{00A250}
    \definecolor{ansi-green-intense}{HTML}{007427}
    \definecolor{ansi-yellow}{HTML}{DDB62B}
    \definecolor{ansi-yellow-intense}{HTML}{B27D12}
    \definecolor{ansi-blue}{HTML}{208FFB}
    \definecolor{ansi-blue-intense}{HTML}{0065CA}
    \definecolor{ansi-magenta}{HTML}{D160C4}
    \definecolor{ansi-magenta-intense}{HTML}{A03196}
    \definecolor{ansi-cyan}{HTML}{60C6C8}
    \definecolor{ansi-cyan-intense}{HTML}{258F8F}
    \definecolor{ansi-white}{HTML}{C5C1B4}
    \definecolor{ansi-white-intense}{HTML}{A1A6B2}
    \definecolor{ansi-default-inverse-fg}{HTML}{FFFFFF}
    \definecolor{ansi-default-inverse-bg}{HTML}{000000}

    % common color for the border for error outputs.
    \definecolor{outerrorbackground}{HTML}{FFDFDF}

    % commands and environments needed by pandoc snippets
    % extracted from the output of `pandoc -s`
    \providecommand{\tightlist}{%
      \setlength{\itemsep}{0pt}\setlength{\parskip}{0pt}}
    \DefineVerbatimEnvironment{Highlighting}{Verbatim}{commandchars=\\\{\}}
    % Add ',fontsize=\small' for more characters per line
    \newenvironment{Shaded}{}{}
    \newcommand{\KeywordTok}[1]{\textcolor[rgb]{0.00,0.44,0.13}{\textbf{{#1}}}}
    \newcommand{\DataTypeTok}[1]{\textcolor[rgb]{0.56,0.13,0.00}{{#1}}}
    \newcommand{\DecValTok}[1]{\textcolor[rgb]{0.25,0.63,0.44}{{#1}}}
    \newcommand{\BaseNTok}[1]{\textcolor[rgb]{0.25,0.63,0.44}{{#1}}}
    \newcommand{\FloatTok}[1]{\textcolor[rgb]{0.25,0.63,0.44}{{#1}}}
    \newcommand{\CharTok}[1]{\textcolor[rgb]{0.25,0.44,0.63}{{#1}}}
    \newcommand{\StringTok}[1]{\textcolor[rgb]{0.25,0.44,0.63}{{#1}}}
    \newcommand{\CommentTok}[1]{\textcolor[rgb]{0.38,0.63,0.69}{\textit{{#1}}}}
    \newcommand{\OtherTok}[1]{\textcolor[rgb]{0.00,0.44,0.13}{{#1}}}
    \newcommand{\AlertTok}[1]{\textcolor[rgb]{1.00,0.00,0.00}{\textbf{{#1}}}}
    \newcommand{\FunctionTok}[1]{\textcolor[rgb]{0.02,0.16,0.49}{{#1}}}
    \newcommand{\RegionMarkerTok}[1]{{#1}}
    \newcommand{\ErrorTok}[1]{\textcolor[rgb]{1.00,0.00,0.00}{\textbf{{#1}}}}
    \newcommand{\NormalTok}[1]{{#1}}

    % Additional commands for more recent versions of Pandoc
    \newcommand{\ConstantTok}[1]{\textcolor[rgb]{0.53,0.00,0.00}{{#1}}}
    \newcommand{\SpecialCharTok}[1]{\textcolor[rgb]{0.25,0.44,0.63}{{#1}}}
    \newcommand{\VerbatimStringTok}[1]{\textcolor[rgb]{0.25,0.44,0.63}{{#1}}}
    \newcommand{\SpecialStringTok}[1]{\textcolor[rgb]{0.73,0.40,0.53}{{#1}}}
    \newcommand{\ImportTok}[1]{{#1}}
    \newcommand{\DocumentationTok}[1]{\textcolor[rgb]{0.73,0.13,0.13}{\textit{{#1}}}}
    \newcommand{\AnnotationTok}[1]{\textcolor[rgb]{0.38,0.63,0.69}{\textbf{\textit{{#1}}}}}
    \newcommand{\CommentVarTok}[1]{\textcolor[rgb]{0.38,0.63,0.69}{\textbf{\textit{{#1}}}}}
    \newcommand{\VariableTok}[1]{\textcolor[rgb]{0.10,0.09,0.49}{{#1}}}
    \newcommand{\ControlFlowTok}[1]{\textcolor[rgb]{0.00,0.44,0.13}{\textbf{{#1}}}}
    \newcommand{\OperatorTok}[1]{\textcolor[rgb]{0.40,0.40,0.40}{{#1}}}
    \newcommand{\BuiltInTok}[1]{{#1}}
    \newcommand{\ExtensionTok}[1]{{#1}}
    \newcommand{\PreprocessorTok}[1]{\textcolor[rgb]{0.74,0.48,0.00}{{#1}}}
    \newcommand{\AttributeTok}[1]{\textcolor[rgb]{0.49,0.56,0.16}{{#1}}}
    \newcommand{\InformationTok}[1]{\textcolor[rgb]{0.38,0.63,0.69}{\textbf{\textit{{#1}}}}}
    \newcommand{\WarningTok}[1]{\textcolor[rgb]{0.38,0.63,0.69}{\textbf{\textit{{#1}}}}}


    % Define a nice break command that doesn't care if a line doesn't already
    % exist.
    \def\br{\hspace*{\fill} \\* }
    % Math Jax compatibility definitions
    \def\gt{>}
    \def\lt{<}
    \let\Oldtex\TeX
    \let\Oldlatex\LaTeX
    \renewcommand{\TeX}{\textrm{\Oldtex}}
    \renewcommand{\LaTeX}{\textrm{\Oldlatex}}
    % Document parameters
    % Document title
    \title{EDA}
    
    
    
    
    
% Pygments definitions
\makeatletter
\def\PY@reset{\let\PY@it=\relax \let\PY@bf=\relax%
    \let\PY@ul=\relax \let\PY@tc=\relax%
    \let\PY@bc=\relax \let\PY@ff=\relax}
\def\PY@tok#1{\csname PY@tok@#1\endcsname}
\def\PY@toks#1+{\ifx\relax#1\empty\else%
    \PY@tok{#1}\expandafter\PY@toks\fi}
\def\PY@do#1{\PY@bc{\PY@tc{\PY@ul{%
    \PY@it{\PY@bf{\PY@ff{#1}}}}}}}
\def\PY#1#2{\PY@reset\PY@toks#1+\relax+\PY@do{#2}}

\@namedef{PY@tok@w}{\def\PY@tc##1{\textcolor[rgb]{0.73,0.73,0.73}{##1}}}
\@namedef{PY@tok@c}{\let\PY@it=\textit\def\PY@tc##1{\textcolor[rgb]{0.25,0.50,0.50}{##1}}}
\@namedef{PY@tok@cp}{\def\PY@tc##1{\textcolor[rgb]{0.74,0.48,0.00}{##1}}}
\@namedef{PY@tok@k}{\let\PY@bf=\textbf\def\PY@tc##1{\textcolor[rgb]{0.00,0.50,0.00}{##1}}}
\@namedef{PY@tok@kp}{\def\PY@tc##1{\textcolor[rgb]{0.00,0.50,0.00}{##1}}}
\@namedef{PY@tok@kt}{\def\PY@tc##1{\textcolor[rgb]{0.69,0.00,0.25}{##1}}}
\@namedef{PY@tok@o}{\def\PY@tc##1{\textcolor[rgb]{0.40,0.40,0.40}{##1}}}
\@namedef{PY@tok@ow}{\let\PY@bf=\textbf\def\PY@tc##1{\textcolor[rgb]{0.67,0.13,1.00}{##1}}}
\@namedef{PY@tok@nb}{\def\PY@tc##1{\textcolor[rgb]{0.00,0.50,0.00}{##1}}}
\@namedef{PY@tok@nf}{\def\PY@tc##1{\textcolor[rgb]{0.00,0.00,1.00}{##1}}}
\@namedef{PY@tok@nc}{\let\PY@bf=\textbf\def\PY@tc##1{\textcolor[rgb]{0.00,0.00,1.00}{##1}}}
\@namedef{PY@tok@nn}{\let\PY@bf=\textbf\def\PY@tc##1{\textcolor[rgb]{0.00,0.00,1.00}{##1}}}
\@namedef{PY@tok@ne}{\let\PY@bf=\textbf\def\PY@tc##1{\textcolor[rgb]{0.82,0.25,0.23}{##1}}}
\@namedef{PY@tok@nv}{\def\PY@tc##1{\textcolor[rgb]{0.10,0.09,0.49}{##1}}}
\@namedef{PY@tok@no}{\def\PY@tc##1{\textcolor[rgb]{0.53,0.00,0.00}{##1}}}
\@namedef{PY@tok@nl}{\def\PY@tc##1{\textcolor[rgb]{0.63,0.63,0.00}{##1}}}
\@namedef{PY@tok@ni}{\let\PY@bf=\textbf\def\PY@tc##1{\textcolor[rgb]{0.60,0.60,0.60}{##1}}}
\@namedef{PY@tok@na}{\def\PY@tc##1{\textcolor[rgb]{0.49,0.56,0.16}{##1}}}
\@namedef{PY@tok@nt}{\let\PY@bf=\textbf\def\PY@tc##1{\textcolor[rgb]{0.00,0.50,0.00}{##1}}}
\@namedef{PY@tok@nd}{\def\PY@tc##1{\textcolor[rgb]{0.67,0.13,1.00}{##1}}}
\@namedef{PY@tok@s}{\def\PY@tc##1{\textcolor[rgb]{0.73,0.13,0.13}{##1}}}
\@namedef{PY@tok@sd}{\let\PY@it=\textit\def\PY@tc##1{\textcolor[rgb]{0.73,0.13,0.13}{##1}}}
\@namedef{PY@tok@si}{\let\PY@bf=\textbf\def\PY@tc##1{\textcolor[rgb]{0.73,0.40,0.53}{##1}}}
\@namedef{PY@tok@se}{\let\PY@bf=\textbf\def\PY@tc##1{\textcolor[rgb]{0.73,0.40,0.13}{##1}}}
\@namedef{PY@tok@sr}{\def\PY@tc##1{\textcolor[rgb]{0.73,0.40,0.53}{##1}}}
\@namedef{PY@tok@ss}{\def\PY@tc##1{\textcolor[rgb]{0.10,0.09,0.49}{##1}}}
\@namedef{PY@tok@sx}{\def\PY@tc##1{\textcolor[rgb]{0.00,0.50,0.00}{##1}}}
\@namedef{PY@tok@m}{\def\PY@tc##1{\textcolor[rgb]{0.40,0.40,0.40}{##1}}}
\@namedef{PY@tok@gh}{\let\PY@bf=\textbf\def\PY@tc##1{\textcolor[rgb]{0.00,0.00,0.50}{##1}}}
\@namedef{PY@tok@gu}{\let\PY@bf=\textbf\def\PY@tc##1{\textcolor[rgb]{0.50,0.00,0.50}{##1}}}
\@namedef{PY@tok@gd}{\def\PY@tc##1{\textcolor[rgb]{0.63,0.00,0.00}{##1}}}
\@namedef{PY@tok@gi}{\def\PY@tc##1{\textcolor[rgb]{0.00,0.63,0.00}{##1}}}
\@namedef{PY@tok@gr}{\def\PY@tc##1{\textcolor[rgb]{1.00,0.00,0.00}{##1}}}
\@namedef{PY@tok@ge}{\let\PY@it=\textit}
\@namedef{PY@tok@gs}{\let\PY@bf=\textbf}
\@namedef{PY@tok@gp}{\let\PY@bf=\textbf\def\PY@tc##1{\textcolor[rgb]{0.00,0.00,0.50}{##1}}}
\@namedef{PY@tok@go}{\def\PY@tc##1{\textcolor[rgb]{0.53,0.53,0.53}{##1}}}
\@namedef{PY@tok@gt}{\def\PY@tc##1{\textcolor[rgb]{0.00,0.27,0.87}{##1}}}
\@namedef{PY@tok@err}{\def\PY@bc##1{{\setlength{\fboxsep}{\string -\fboxrule}\fcolorbox[rgb]{1.00,0.00,0.00}{1,1,1}{\strut ##1}}}}
\@namedef{PY@tok@kc}{\let\PY@bf=\textbf\def\PY@tc##1{\textcolor[rgb]{0.00,0.50,0.00}{##1}}}
\@namedef{PY@tok@kd}{\let\PY@bf=\textbf\def\PY@tc##1{\textcolor[rgb]{0.00,0.50,0.00}{##1}}}
\@namedef{PY@tok@kn}{\let\PY@bf=\textbf\def\PY@tc##1{\textcolor[rgb]{0.00,0.50,0.00}{##1}}}
\@namedef{PY@tok@kr}{\let\PY@bf=\textbf\def\PY@tc##1{\textcolor[rgb]{0.00,0.50,0.00}{##1}}}
\@namedef{PY@tok@bp}{\def\PY@tc##1{\textcolor[rgb]{0.00,0.50,0.00}{##1}}}
\@namedef{PY@tok@fm}{\def\PY@tc##1{\textcolor[rgb]{0.00,0.00,1.00}{##1}}}
\@namedef{PY@tok@vc}{\def\PY@tc##1{\textcolor[rgb]{0.10,0.09,0.49}{##1}}}
\@namedef{PY@tok@vg}{\def\PY@tc##1{\textcolor[rgb]{0.10,0.09,0.49}{##1}}}
\@namedef{PY@tok@vi}{\def\PY@tc##1{\textcolor[rgb]{0.10,0.09,0.49}{##1}}}
\@namedef{PY@tok@vm}{\def\PY@tc##1{\textcolor[rgb]{0.10,0.09,0.49}{##1}}}
\@namedef{PY@tok@sa}{\def\PY@tc##1{\textcolor[rgb]{0.73,0.13,0.13}{##1}}}
\@namedef{PY@tok@sb}{\def\PY@tc##1{\textcolor[rgb]{0.73,0.13,0.13}{##1}}}
\@namedef{PY@tok@sc}{\def\PY@tc##1{\textcolor[rgb]{0.73,0.13,0.13}{##1}}}
\@namedef{PY@tok@dl}{\def\PY@tc##1{\textcolor[rgb]{0.73,0.13,0.13}{##1}}}
\@namedef{PY@tok@s2}{\def\PY@tc##1{\textcolor[rgb]{0.73,0.13,0.13}{##1}}}
\@namedef{PY@tok@sh}{\def\PY@tc##1{\textcolor[rgb]{0.73,0.13,0.13}{##1}}}
\@namedef{PY@tok@s1}{\def\PY@tc##1{\textcolor[rgb]{0.73,0.13,0.13}{##1}}}
\@namedef{PY@tok@mb}{\def\PY@tc##1{\textcolor[rgb]{0.40,0.40,0.40}{##1}}}
\@namedef{PY@tok@mf}{\def\PY@tc##1{\textcolor[rgb]{0.40,0.40,0.40}{##1}}}
\@namedef{PY@tok@mh}{\def\PY@tc##1{\textcolor[rgb]{0.40,0.40,0.40}{##1}}}
\@namedef{PY@tok@mi}{\def\PY@tc##1{\textcolor[rgb]{0.40,0.40,0.40}{##1}}}
\@namedef{PY@tok@il}{\def\PY@tc##1{\textcolor[rgb]{0.40,0.40,0.40}{##1}}}
\@namedef{PY@tok@mo}{\def\PY@tc##1{\textcolor[rgb]{0.40,0.40,0.40}{##1}}}
\@namedef{PY@tok@ch}{\let\PY@it=\textit\def\PY@tc##1{\textcolor[rgb]{0.25,0.50,0.50}{##1}}}
\@namedef{PY@tok@cm}{\let\PY@it=\textit\def\PY@tc##1{\textcolor[rgb]{0.25,0.50,0.50}{##1}}}
\@namedef{PY@tok@cpf}{\let\PY@it=\textit\def\PY@tc##1{\textcolor[rgb]{0.25,0.50,0.50}{##1}}}
\@namedef{PY@tok@c1}{\let\PY@it=\textit\def\PY@tc##1{\textcolor[rgb]{0.25,0.50,0.50}{##1}}}
\@namedef{PY@tok@cs}{\let\PY@it=\textit\def\PY@tc##1{\textcolor[rgb]{0.25,0.50,0.50}{##1}}}

\def\PYZbs{\char`\\}
\def\PYZus{\char`\_}
\def\PYZob{\char`\{}
\def\PYZcb{\char`\}}
\def\PYZca{\char`\^}
\def\PYZam{\char`\&}
\def\PYZlt{\char`\<}
\def\PYZgt{\char`\>}
\def\PYZsh{\char`\#}
\def\PYZpc{\char`\%}
\def\PYZdl{\char`\$}
\def\PYZhy{\char`\-}
\def\PYZsq{\char`\'}
\def\PYZdq{\char`\"}
\def\PYZti{\char`\~}
% for compatibility with earlier versions
\def\PYZat{@}
\def\PYZlb{[}
\def\PYZrb{]}
\makeatother


    % For linebreaks inside Verbatim environment from package fancyvrb.
    \makeatletter
        \newbox\Wrappedcontinuationbox
        \newbox\Wrappedvisiblespacebox
        \newcommand*\Wrappedvisiblespace {\textcolor{red}{\textvisiblespace}}
        \newcommand*\Wrappedcontinuationsymbol {\textcolor{red}{\llap{\tiny$\m@th\hookrightarrow$}}}
        \newcommand*\Wrappedcontinuationindent {3ex }
        \newcommand*\Wrappedafterbreak {\kern\Wrappedcontinuationindent\copy\Wrappedcontinuationbox}
        % Take advantage of the already applied Pygments mark-up to insert
        % potential linebreaks for TeX processing.
        %        {, <, #, %, $, ' and ": go to next line.
        %        _, }, ^, &, >, - and ~: stay at end of broken line.
        % Use of \textquotesingle for straight quote.
        \newcommand*\Wrappedbreaksatspecials {%
            \def\PYGZus{\discretionary{\char`\_}{\Wrappedafterbreak}{\char`\_}}%
            \def\PYGZob{\discretionary{}{\Wrappedafterbreak\char`\{}{\char`\{}}%
            \def\PYGZcb{\discretionary{\char`\}}{\Wrappedafterbreak}{\char`\}}}%
            \def\PYGZca{\discretionary{\char`\^}{\Wrappedafterbreak}{\char`\^}}%
            \def\PYGZam{\discretionary{\char`\&}{\Wrappedafterbreak}{\char`\&}}%
            \def\PYGZlt{\discretionary{}{\Wrappedafterbreak\char`\<}{\char`\<}}%
            \def\PYGZgt{\discretionary{\char`\>}{\Wrappedafterbreak}{\char`\>}}%
            \def\PYGZsh{\discretionary{}{\Wrappedafterbreak\char`\#}{\char`\#}}%
            \def\PYGZpc{\discretionary{}{\Wrappedafterbreak\char`\%}{\char`\%}}%
            \def\PYGZdl{\discretionary{}{\Wrappedafterbreak\char`\$}{\char`\$}}%
            \def\PYGZhy{\discretionary{\char`\-}{\Wrappedafterbreak}{\char`\-}}%
            \def\PYGZsq{\discretionary{}{\Wrappedafterbreak\textquotesingle}{\textquotesingle}}%
            \def\PYGZdq{\discretionary{}{\Wrappedafterbreak\char`\"}{\char`\"}}%
            \def\PYGZti{\discretionary{\char`\~}{\Wrappedafterbreak}{\char`\~}}%
        }
        % Some characters . , ; ? ! / are not pygmentized.
        % This macro makes them "active" and they will insert potential linebreaks
        \newcommand*\Wrappedbreaksatpunct {%
            \lccode`\~`\.\lowercase{\def~}{\discretionary{\hbox{\char`\.}}{\Wrappedafterbreak}{\hbox{\char`\.}}}%
            \lccode`\~`\,\lowercase{\def~}{\discretionary{\hbox{\char`\,}}{\Wrappedafterbreak}{\hbox{\char`\,}}}%
            \lccode`\~`\;\lowercase{\def~}{\discretionary{\hbox{\char`\;}}{\Wrappedafterbreak}{\hbox{\char`\;}}}%
            \lccode`\~`\:\lowercase{\def~}{\discretionary{\hbox{\char`\:}}{\Wrappedafterbreak}{\hbox{\char`\:}}}%
            \lccode`\~`\?\lowercase{\def~}{\discretionary{\hbox{\char`\?}}{\Wrappedafterbreak}{\hbox{\char`\?}}}%
            \lccode`\~`\!\lowercase{\def~}{\discretionary{\hbox{\char`\!}}{\Wrappedafterbreak}{\hbox{\char`\!}}}%
            \lccode`\~`\/\lowercase{\def~}{\discretionary{\hbox{\char`\/}}{\Wrappedafterbreak}{\hbox{\char`\/}}}%
            \catcode`\.\active
            \catcode`\,\active
            \catcode`\;\active
            \catcode`\:\active
            \catcode`\?\active
            \catcode`\!\active
            \catcode`\/\active
            \lccode`\~`\~
        }
    \makeatother

    \let\OriginalVerbatim=\Verbatim
    \makeatletter
    \renewcommand{\Verbatim}[1][1]{%
        %\parskip\z@skip
        \sbox\Wrappedcontinuationbox {\Wrappedcontinuationsymbol}%
        \sbox\Wrappedvisiblespacebox {\FV@SetupFont\Wrappedvisiblespace}%
        \def\FancyVerbFormatLine ##1{\hsize\linewidth
            \vtop{\raggedright\hyphenpenalty\z@\exhyphenpenalty\z@
                \doublehyphendemerits\z@\finalhyphendemerits\z@
                \strut ##1\strut}%
        }%
        % If the linebreak is at a space, the latter will be displayed as visible
        % space at end of first line, and a continuation symbol starts next line.
        % Stretch/shrink are however usually zero for typewriter font.
        \def\FV@Space {%
            \nobreak\hskip\z@ plus\fontdimen3\font minus\fontdimen4\font
            \discretionary{\copy\Wrappedvisiblespacebox}{\Wrappedafterbreak}
            {\kern\fontdimen2\font}%
        }%

        % Allow breaks at special characters using \PYG... macros.
        \Wrappedbreaksatspecials
        % Breaks at punctuation characters . , ; ? ! and / need catcode=\active
        \OriginalVerbatim[#1,codes*=\Wrappedbreaksatpunct]%
    }
    \makeatother

    % Exact colors from NB
    \definecolor{incolor}{HTML}{303F9F}
    \definecolor{outcolor}{HTML}{D84315}
    \definecolor{cellborder}{HTML}{CFCFCF}
    \definecolor{cellbackground}{HTML}{F7F7F7}

    % prompt
    \makeatletter
    \newcommand{\boxspacing}{\kern\kvtcb@left@rule\kern\kvtcb@boxsep}
    \makeatother
    \newcommand{\prompt}[4]{
        {\ttfamily\llap{{\color{#2}[#3]:\hspace{3pt}#4}}\vspace{-\baselineskip}}
    }
    

    
    % Prevent overflowing lines due to hard-to-break entities
    \sloppy
    % Setup hyperref package
    \hypersetup{
      breaklinks=true,  % so long urls are correctly broken across lines
      colorlinks=true,
      urlcolor=urlcolor,
      linkcolor=linkcolor,
      citecolor=citecolor,
      }
    % Slightly bigger margins than the latex defaults
    
    \geometry{verbose,tmargin=1in,bmargin=1in,lmargin=1in,rmargin=1in}
    
    

\begin{document}
    
    % \maketitle
    
    

    
    \begin{tcolorbox}[breakable, size=fbox, boxrule=1pt, pad at break*=1mm,colback=cellbackground, colframe=cellborder]
\prompt{In}{incolor}{1}{\boxspacing}
\begin{Verbatim}[commandchars=\\\{\}]
\PY{c+c1}{\PYZsh{} MSDS 422 \PYZhy{} Section 55}
\PY{c+c1}{\PYZsh{} Spring \PYZsq{}24}
\PY{c+c1}{\PYZsh{} Module 05 \PYZhy{} Midpoint check in for final project}

\PY{c+c1}{\PYZsh{} Initial EDA}
\PY{c+c1}{\PYZsh{} Kevin Geidel}

\PY{k+kn}{import} \PY{n+nn}{numpy} \PY{k}{as} \PY{n+nn}{np}
\PY{k+kn}{import} \PY{n+nn}{pandas} \PY{k}{as} \PY{n+nn}{pd}
\PY{k+kn}{import} \PY{n+nn}{os}
\PY{k+kn}{import} \PY{n+nn}{settings}         \PY{c+c1}{\PYZsh{} runs commands that sets base paths, configures behavors, etc}
\PY{k+kn}{import} \PY{n+nn}{utils}            \PY{c+c1}{\PYZsh{} defines functions used throughout}
\end{Verbatim}
\end{tcolorbox}

    \hypertarget{eda-data-source}{%
\paragraph{EDA: Data source}\label{eda-data-source}}

    \begin{tcolorbox}[breakable, size=fbox, boxrule=1pt, pad at break*=1mm,colback=cellbackground, colframe=cellborder]
\prompt{In}{incolor}{2}{\boxspacing}
\begin{Verbatim}[commandchars=\\\{\}]
\PY{c+c1}{\PYZsh{} Load the dataset}

\PY{k+kn}{from} \PY{n+nn}{scipy}\PY{n+nn}{.}\PY{n+nn}{io}\PY{n+nn}{.}\PY{n+nn}{arff} \PY{k+kn}{import} \PY{n}{loadarff}

\PY{n}{arff\PYZus{}name} \PY{o}{=} \PY{l+s+s1}{\PYZsq{}}\PY{l+s+s1}{wine.arff}\PY{l+s+s1}{\PYZsq{}}

\PY{n}{raw\PYZus{}data} \PY{o}{=} \PY{n}{loadarff}\PY{p}{(}
    \PY{n}{os}\PY{o}{.}\PY{n}{path}\PY{o}{.}\PY{n}{join}\PY{p}{(}\PY{n}{settings}\PY{o}{.}\PY{n}{DATA\PYZus{}PATH}\PY{p}{,} \PY{n}{arff\PYZus{}name}\PY{p}{)}
\PY{p}{)}
\PY{n}{data} \PY{o}{=} \PY{n}{pd}\PY{o}{.}\PY{n}{DataFrame}\PY{p}{(}\PY{n}{raw\PYZus{}data}\PY{p}{[}\PY{l+m+mi}{0}\PY{p}{]}\PY{p}{)}

\PY{n}{data}\PY{o}{.}\PY{n}{head}\PY{p}{(}\PY{p}{)}
\end{Verbatim}
\end{tcolorbox}

            \begin{tcolorbox}[breakable, size=fbox, boxrule=.5pt, pad at break*=1mm, opacityfill=0]
\prompt{Out}{outcolor}{2}{\boxspacing}
\begin{Verbatim}[commandchars=\\\{\}]
   Alcohol  Malic\_acid   Ash  Alcalinity\_of\_ash  Magnesium  Total\_phenols  \textbackslash{}
0    14.23        1.71  2.43               15.6      127.0           2.80
1    13.20        1.78  2.14               11.2      100.0           2.65
2    13.16        2.36  2.67               18.6      101.0           2.80
3    14.37        1.95  2.50               16.8      113.0           3.85
4    13.24        2.59  2.87               21.0      118.0           2.80

   Flavanoids  Nonflavanoid\_phenols  Proanthocyanins  Color\_intensity   Hue  \textbackslash{}
0        3.06                  0.28             2.29             5.64  1.04
1        2.76                  0.26             1.28             4.38  1.05
2        3.24                  0.30             2.81             5.68  1.03
3        3.49                  0.24             2.18             7.80  0.86
4        2.69                  0.39             1.82             4.32  1.04

   OD280/OD315\_of\_diluted\_wines  Proline binaryClass
0                          3.92   1065.0        b'N'
1                          3.40   1050.0        b'N'
2                          3.17   1185.0        b'N'
3                          3.45   1480.0        b'N'
4                          2.93    735.0        b'N'
\end{Verbatim}
\end{tcolorbox}
        
    \begin{tcolorbox}[breakable, size=fbox, boxrule=1pt, pad at break*=1mm,colback=cellbackground, colframe=cellborder]
\prompt{In}{incolor}{3}{\boxspacing}
\begin{Verbatim}[commandchars=\\\{\}]
\PY{c+c1}{\PYZsh{} Examine the dataset}

\PY{n}{data}\PY{o}{.}\PY{n}{info}\PY{p}{(}\PY{p}{)}
\end{Verbatim}
\end{tcolorbox}

    \begin{Verbatim}[commandchars=\\\{\}]
<class 'pandas.core.frame.DataFrame'>
RangeIndex: 178 entries, 0 to 177
Data columns (total 14 columns):
 \#   Column                        Non-Null Count  Dtype
---  ------                        --------------  -----
 0   Alcohol                       178 non-null    float64
 1   Malic\_acid                    178 non-null    float64
 2   Ash                           178 non-null    float64
 3   Alcalinity\_of\_ash             178 non-null    float64
 4   Magnesium                     178 non-null    float64
 5   Total\_phenols                 178 non-null    float64
 6   Flavanoids                    178 non-null    float64
 7   Nonflavanoid\_phenols          178 non-null    float64
 8   Proanthocyanins               178 non-null    float64
 9   Color\_intensity               178 non-null    float64
 10  Hue                           178 non-null    float64
 11  OD280/OD315\_of\_diluted\_wines  178 non-null    float64
 12  Proline                       178 non-null    float64
 13  binaryClass                   178 non-null    object
dtypes: float64(13), object(1)
memory usage: 19.6+ KB
    \end{Verbatim}

    \begin{tcolorbox}[breakable, size=fbox, boxrule=1pt, pad at break*=1mm,colback=cellbackground, colframe=cellborder]
\prompt{In}{incolor}{4}{\boxspacing}
\begin{Verbatim}[commandchars=\\\{\}]
\PY{c+c1}{\PYZsh{} The target (binaryClass) happens to be the only caterorical (in this case Boolean) variable}
\PY{c+c1}{\PYZsh{} but still, we look at each one}

\PY{n}{data}\PY{o}{.}\PY{n}{select\PYZus{}dtypes}\PY{p}{(}\PY{n}{include}\PY{o}{=}\PY{p}{[}\PY{n+nb}{object}\PY{p}{]}\PY{p}{)}\PY{o}{.}\PY{n}{value\PYZus{}counts}\PY{p}{(}\PY{p}{)}
\end{Verbatim}
\end{tcolorbox}

            \begin{tcolorbox}[breakable, size=fbox, boxrule=.5pt, pad at break*=1mm, opacityfill=0]
\prompt{Out}{outcolor}{4}{\boxspacing}
\begin{Verbatim}[commandchars=\\\{\}]
binaryClass
b'N'           107
b'P'            71
Name: count, dtype: int64
\end{Verbatim}
\end{tcolorbox}
        
    \begin{tcolorbox}[breakable, size=fbox, boxrule=1pt, pad at break*=1mm,colback=cellbackground, colframe=cellborder]
\prompt{In}{incolor}{5}{\boxspacing}
\begin{Verbatim}[commandchars=\\\{\}]
\PY{c+c1}{\PYZsh{} We are going to encode the boolean values later anyways}
\PY{c+c1}{\PYZsh{} let\PYZsq{}s address this now so we can use the target in the rest of the EDA}

\PY{n}{data}\PY{p}{[}\PY{l+s+s1}{\PYZsq{}}\PY{l+s+s1}{isWine}\PY{l+s+s1}{\PYZsq{}}\PY{p}{]} \PY{o}{=} \PY{n}{data}\PY{p}{[}\PY{l+s+s1}{\PYZsq{}}\PY{l+s+s1}{binaryClass}\PY{l+s+s1}{\PYZsq{}}\PY{p}{]}\PY{o}{.}\PY{n}{replace}\PY{p}{(}\PY{p}{\PYZob{}}\PY{l+s+sa}{b}\PY{l+s+s1}{\PYZsq{}}\PY{l+s+s1}{N}\PY{l+s+s1}{\PYZsq{}}\PY{p}{:} \PY{l+m+mi}{0}\PY{p}{,} \PY{l+s+sa}{b}\PY{l+s+s1}{\PYZsq{}}\PY{l+s+s1}{P}\PY{l+s+s1}{\PYZsq{}}\PY{p}{:} \PY{l+m+mi}{1}\PY{p}{\PYZcb{}}\PY{p}{)}

\PY{n}{data}\PY{o}{.}\PY{n}{head}\PY{p}{(}\PY{p}{)}
\end{Verbatim}
\end{tcolorbox}

            \begin{tcolorbox}[breakable, size=fbox, boxrule=.5pt, pad at break*=1mm, opacityfill=0]
\prompt{Out}{outcolor}{5}{\boxspacing}
\begin{Verbatim}[commandchars=\\\{\}]
   Alcohol  Malic\_acid   Ash  Alcalinity\_of\_ash  Magnesium  Total\_phenols  \textbackslash{}
0    14.23        1.71  2.43               15.6      127.0           2.80
1    13.20        1.78  2.14               11.2      100.0           2.65
2    13.16        2.36  2.67               18.6      101.0           2.80
3    14.37        1.95  2.50               16.8      113.0           3.85
4    13.24        2.59  2.87               21.0      118.0           2.80

   Flavanoids  Nonflavanoid\_phenols  Proanthocyanins  Color\_intensity   Hue  \textbackslash{}
0        3.06                  0.28             2.29             5.64  1.04
1        2.76                  0.26             1.28             4.38  1.05
2        3.24                  0.30             2.81             5.68  1.03
3        3.49                  0.24             2.18             7.80  0.86
4        2.69                  0.39             1.82             4.32  1.04

   OD280/OD315\_of\_diluted\_wines  Proline binaryClass  isWine
0                          3.92   1065.0        b'N'       0
1                          3.40   1050.0        b'N'       0
2                          3.17   1185.0        b'N'       0
3                          3.45   1480.0        b'N'       0
4                          2.93    735.0        b'N'       0
\end{Verbatim}
\end{tcolorbox}
        
    \hypertarget{eda-basic-data-structure}{%
\paragraph{EDA: Basic data structure}\label{eda-basic-data-structure}}

    \begin{tcolorbox}[breakable, size=fbox, boxrule=1pt, pad at break*=1mm,colback=cellbackground, colframe=cellborder]
\prompt{In}{incolor}{6}{\boxspacing}
\begin{Verbatim}[commandchars=\\\{\}]
\PY{c+c1}{\PYZsh{} Descriptive stats of numerical columns}

\PY{n}{data}\PY{o}{.}\PY{n}{describe}\PY{p}{(}\PY{p}{)}
\end{Verbatim}
\end{tcolorbox}

            \begin{tcolorbox}[breakable, size=fbox, boxrule=.5pt, pad at break*=1mm, opacityfill=0]
\prompt{Out}{outcolor}{6}{\boxspacing}
\begin{Verbatim}[commandchars=\\\{\}]
          Alcohol  Malic\_acid         Ash  Alcalinity\_of\_ash   Magnesium  \textbackslash{}
count  178.000000  178.000000  178.000000         178.000000  178.000000
mean    13.000618    2.336348    2.366517          19.494944   99.741573
std      0.811827    1.117146    0.274344           3.339564   14.282484
min     11.030000    0.740000    1.360000          10.600000   70.000000
25\%     12.362500    1.602500    2.210000          17.200000   88.000000
50\%     13.050000    1.865000    2.360000          19.500000   98.000000
75\%     13.677500    3.082500    2.557500          21.500000  107.000000
max     14.830000    5.800000    3.230000          30.000000  162.000000

       Total\_phenols  Flavanoids  Nonflavanoid\_phenols  Proanthocyanins  \textbackslash{}
count     178.000000  178.000000            178.000000       178.000000
mean        2.295112    2.029270              0.361854         1.590899
std         0.625851    0.998859              0.124453         0.572359
min         0.980000    0.340000              0.130000         0.410000
25\%         1.742500    1.205000              0.270000         1.250000
50\%         2.355000    2.135000              0.340000         1.555000
75\%         2.800000    2.875000              0.437500         1.950000
max         3.880000    5.080000              0.660000         3.580000

       Color\_intensity         Hue  OD280/OD315\_of\_diluted\_wines      Proline  \textbackslash{}
count       178.000000  178.000000                    178.000000   178.000000
mean          5.058090    0.957449                      2.611685   746.893258
std           2.318286    0.228572                      0.709990   314.907474
min           1.280000    0.480000                      1.270000   278.000000
25\%           3.220000    0.782500                      1.937500   500.500000
50\%           4.690000    0.965000                      2.780000   673.500000
75\%           6.200000    1.120000                      3.170000   985.000000
max          13.000000    1.710000                      4.000000  1680.000000

           isWine
count  178.000000
mean     0.398876
std      0.491049
min      0.000000
25\%      0.000000
50\%      0.000000
75\%      1.000000
max      1.000000
\end{Verbatim}
\end{tcolorbox}
        
    \begin{tcolorbox}[breakable, size=fbox, boxrule=1pt, pad at break*=1mm,colback=cellbackground, colframe=cellborder]
\prompt{In}{incolor}{7}{\boxspacing}
\begin{Verbatim}[commandchars=\\\{\}]
\PY{c+c1}{\PYZsh{} Get null counts}

\PY{n}{pd}\PY{o}{.}\PY{n}{DataFrame}\PY{p}{(}
    \PY{p}{[}\PY{p}{(}\PY{n}{col}\PY{p}{,} \PY{n}{data}\PY{p}{[}\PY{n}{col}\PY{p}{]}\PY{o}{.}\PY{n}{isnull}\PY{p}{(}\PY{p}{)}\PY{o}{.}\PY{n}{sum}\PY{p}{(}\PY{p}{)}\PY{p}{)} \PY{k}{for} \PY{n}{col} \PY{o+ow}{in} \PY{n}{data}\PY{o}{.}\PY{n}{columns}\PY{p}{]}\PY{p}{,} 
    \PY{n}{columns} \PY{o}{=} \PY{p}{[}\PY{l+s+s1}{\PYZsq{}}\PY{l+s+s1}{Columns Name}\PY{l+s+s1}{\PYZsq{}}\PY{p}{,} \PY{l+s+s1}{\PYZsq{}}\PY{l+s+s1}{Null Count}\PY{l+s+s1}{\PYZsq{}}\PY{p}{]}
\PY{p}{)}
\end{Verbatim}
\end{tcolorbox}

            \begin{tcolorbox}[breakable, size=fbox, boxrule=.5pt, pad at break*=1mm, opacityfill=0]
\prompt{Out}{outcolor}{7}{\boxspacing}
\begin{Verbatim}[commandchars=\\\{\}]
                    Columns Name  Null Count
0                        Alcohol           0
1                     Malic\_acid           0
2                            Ash           0
3              Alcalinity\_of\_ash           0
4                      Magnesium           0
5                  Total\_phenols           0
6                     Flavanoids           0
7           Nonflavanoid\_phenols           0
8                Proanthocyanins           0
9                Color\_intensity           0
10                           Hue           0
11  OD280/OD315\_of\_diluted\_wines           0
12                       Proline           0
13                   binaryClass           0
14                        isWine           0
\end{Verbatim}
\end{tcolorbox}
        
    \hypertarget{eda-initial-visualizations}{%
\paragraph{EDA: Initial
visualizations}\label{eda-initial-visualizations}}

    \begin{tcolorbox}[breakable, size=fbox, boxrule=1pt, pad at break*=1mm,colback=cellbackground, colframe=cellborder]
\prompt{In}{incolor}{8}{\boxspacing}
\begin{Verbatim}[commandchars=\\\{\}]
\PY{c+c1}{\PYZsh{} Make historgrams of the numeric variables for initial visualization}

\PY{k+kn}{import} \PY{n+nn}{matplotlib}\PY{n+nn}{.}\PY{n+nn}{pyplot} \PY{k}{as} \PY{n+nn}{plt}

\PY{n}{data}\PY{o}{.}\PY{n}{hist}\PY{p}{(}\PY{n}{bins}\PY{o}{=}\PY{l+m+mi}{50}\PY{p}{,} \PY{n}{figsize}\PY{o}{=}\PY{p}{(}\PY{l+m+mi}{12}\PY{p}{,} \PY{l+m+mi}{8}\PY{p}{)}\PY{p}{)}

\PY{n}{utils}\PY{o}{.}\PY{n}{save\PYZus{}fig}\PY{p}{(}\PY{l+s+s2}{\PYZdq{}}\PY{l+s+s2}{attribute\PYZus{}histogram\PYZus{}plots}\PY{l+s+s2}{\PYZdq{}}\PY{p}{)}
\PY{n}{plt}\PY{o}{.}\PY{n}{show}\PY{p}{(}\PY{p}{)}
\end{Verbatim}
\end{tcolorbox}

    \begin{center}
    \adjustimage{max size={0.9\linewidth}{0.9\paperheight}}{EDA_files/EDA_10_0.png}
    \end{center}
    { \hspace*{\fill} \\}
    
    \begin{tcolorbox}[breakable, size=fbox, boxrule=1pt, pad at break*=1mm,colback=cellbackground, colframe=cellborder]
\prompt{In}{incolor}{9}{\boxspacing}
\begin{Verbatim}[commandchars=\\\{\}]
\PY{c+c1}{\PYZsh{} Check for linear correlations}

\PY{n}{correlations} \PY{o}{=} \PY{n}{data}\PY{o}{.}\PY{n}{corr}\PY{p}{(}\PY{n}{numeric\PYZus{}only}\PY{o}{=}\PY{k+kc}{True}\PY{p}{)}
\PY{n}{correlations}\PY{p}{[}\PY{l+s+s2}{\PYZdq{}}\PY{l+s+s2}{isWine}\PY{l+s+s2}{\PYZdq{}}\PY{p}{]}\PY{o}{.}\PY{n}{sort\PYZus{}values}\PY{p}{(}\PY{n}{ascending}\PY{o}{=}\PY{k+kc}{False}\PY{p}{)}
\end{Verbatim}
\end{tcolorbox}

            \begin{tcolorbox}[breakable, size=fbox, boxrule=.5pt, pad at break*=1mm, opacityfill=0]
\prompt{Out}{outcolor}{9}{\boxspacing}
\begin{Verbatim}[commandchars=\\\{\}]
isWine                          1.000000
Hue                             0.353213
OD280/OD315\_of\_diluted\_wines    0.199813
Alcalinity\_of\_ash               0.181764
Proanthocyanins                 0.056208
Flavanoids                      0.042179
Nonflavanoid\_phenols            0.011868
Total\_phenols                  -0.047301
Malic\_acid                     -0.295175
Magnesium                      -0.296972
Ash                            -0.362457
Proline                        -0.589850
Color\_intensity                -0.694679
Alcohol                        -0.726383
Name: isWine, dtype: float64
\end{Verbatim}
\end{tcolorbox}
        
    \begin{tcolorbox}[breakable, size=fbox, boxrule=1pt, pad at break*=1mm,colback=cellbackground, colframe=cellborder]
\prompt{In}{incolor}{10}{\boxspacing}
\begin{Verbatim}[commandchars=\\\{\}]
\PY{c+c1}{\PYZsh{} plot scatter plots for interesting columns with promising coefficieients}

\PY{k+kn}{from} \PY{n+nn}{pandas}\PY{n+nn}{.}\PY{n+nn}{plotting} \PY{k+kn}{import} \PY{n}{scatter\PYZus{}matrix}

\PY{n}{scatter\PYZus{}plot\PYZus{}cols} \PY{o}{=} \PY{p}{[}
    \PY{l+s+s1}{\PYZsq{}}\PY{l+s+s1}{isWine}\PY{l+s+s1}{\PYZsq{}}\PY{p}{,} \PY{l+s+s1}{\PYZsq{}}\PY{l+s+s1}{Hue}\PY{l+s+s1}{\PYZsq{}}\PY{p}{,} \PY{l+s+s1}{\PYZsq{}}\PY{l+s+s1}{OD280/OD315\PYZus{}of\PYZus{}diluted\PYZus{}wines}\PY{l+s+s1}{\PYZsq{}}\PY{p}{,} \PY{l+s+s1}{\PYZsq{}}\PY{l+s+s1}{Alcalinity\PYZus{}of\PYZus{}ash}\PY{l+s+s1}{\PYZsq{}}\PY{p}{,}
    \PY{l+s+s1}{\PYZsq{}}\PY{l+s+s1}{Alcohol}\PY{l+s+s1}{\PYZsq{}}\PY{p}{,} \PY{l+s+s1}{\PYZsq{}}\PY{l+s+s1}{Color\PYZus{}intensity}\PY{l+s+s1}{\PYZsq{}}\PY{p}{,} \PY{l+s+s1}{\PYZsq{}}\PY{l+s+s1}{Nonflavanoid\PYZus{}phenols}\PY{l+s+s1}{\PYZsq{}}\PY{p}{,} \PY{l+s+s1}{\PYZsq{}}\PY{l+s+s1}{Flavanoids}\PY{l+s+s1}{\PYZsq{}}
\PY{p}{]}

\PY{n}{scatter\PYZus{}matrix}\PY{p}{(}\PY{n}{data}\PY{p}{[}\PY{n}{scatter\PYZus{}plot\PYZus{}cols}\PY{p}{]}\PY{p}{,} \PY{n}{figsize}\PY{o}{=}\PY{p}{(}\PY{l+m+mi}{12}\PY{p}{,} \PY{l+m+mi}{8}\PY{p}{)}\PY{p}{)}
\PY{n}{utils}\PY{o}{.}\PY{n}{save\PYZus{}fig}\PY{p}{(}\PY{l+s+s2}{\PYZdq{}}\PY{l+s+s2}{scatter\PYZus{}matrix\PYZus{}plot}\PY{l+s+s2}{\PYZdq{}}\PY{p}{)} 
\PY{n}{plt}\PY{o}{.}\PY{n}{show}\PY{p}{(}\PY{p}{)}
\end{Verbatim}
\end{tcolorbox}

    \begin{center}
    \adjustimage{max size={0.9\linewidth}{0.9\paperheight}}{EDA_files/EDA_12_0.png}
    \end{center}
    { \hspace*{\fill} \\}
    
    \begin{tcolorbox}[breakable, size=fbox, boxrule=1pt, pad at break*=1mm,colback=cellbackground, colframe=cellborder]
\prompt{In}{incolor}{43}{\boxspacing}
\begin{Verbatim}[commandchars=\\\{\}]
\PY{c+c1}{\PYZsh{} lets examine some graphs up close}
\PY{c+c1}{\PYZsh{} we are looking for interesting relationships that may not be linear}
\PY{c+c1}{\PYZsh{} and/or may make for good engineered features.}

\PY{n}{single\PYZus{}scatter\PYZus{}ls} \PY{o}{=} \PY{p}{[}
    \PY{p}{(}\PY{l+s+s1}{\PYZsq{}}\PY{l+s+s1}{Color\PYZus{}intensity}\PY{l+s+s1}{\PYZsq{}}\PY{p}{,} \PY{l+s+s1}{\PYZsq{}}\PY{l+s+s1}{Flavanoids}\PY{l+s+s1}{\PYZsq{}}\PY{p}{)}\PY{p}{,} \PY{c+c1}{\PYZsh{} seems to have two distinct groupings}
    \PY{p}{(}\PY{l+s+s1}{\PYZsq{}}\PY{l+s+s1}{Alcohol}\PY{l+s+s1}{\PYZsq{}}\PY{p}{,} \PY{l+s+s1}{\PYZsq{}}\PY{l+s+s1}{Color\PYZus{}intensity}\PY{l+s+s1}{\PYZsq{}}\PY{p}{)}\PY{p}{,}    \PY{c+c1}{\PYZsh{} has very curious shape}
    \PY{p}{(}\PY{l+s+s1}{\PYZsq{}}\PY{l+s+s1}{Alcohol}\PY{l+s+s1}{\PYZsq{}}\PY{p}{,} \PY{l+s+s1}{\PYZsq{}}\PY{l+s+s1}{Flavanoids}\PY{l+s+s1}{\PYZsq{}}\PY{p}{)}\PY{p}{,}         \PY{c+c1}{\PYZsh{} also has a hook shape w/ ostensibly distinct clusters}
\PY{p}{]}

\PY{k}{for} \PY{n}{y}\PY{p}{,} \PY{n}{x} \PY{o+ow}{in} \PY{n}{single\PYZus{}scatter\PYZus{}ls}\PY{p}{:}
    \PY{n}{data}\PY{o}{.}\PY{n}{plot}\PY{p}{(}\PY{n}{kind}\PY{o}{=}\PY{l+s+s1}{\PYZsq{}}\PY{l+s+s1}{scatter}\PY{l+s+s1}{\PYZsq{}}\PY{p}{,} \PY{n}{x}\PY{o}{=}\PY{n}{x}\PY{p}{,} \PY{n}{y}\PY{o}{=}\PY{n}{y}\PY{p}{)}
    \PY{n}{utils}\PY{o}{.}\PY{n}{save\PYZus{}fig}\PY{p}{(}\PY{l+s+sa}{f}\PY{l+s+s2}{\PYZdq{}}\PY{l+s+s2}{scatterplot\PYZus{}}\PY{l+s+si}{\PYZob{}}\PY{n}{y}\PY{l+s+si}{\PYZcb{}}\PY{l+s+s2}{\PYZus{}over\PYZus{}}\PY{l+s+si}{\PYZob{}}\PY{n}{x}\PY{l+s+si}{\PYZcb{}}\PY{l+s+s2}{\PYZdq{}}\PY{p}{)}
    \PY{n}{plt}\PY{o}{.}\PY{n}{show}\PY{p}{(}\PY{p}{)}
\end{Verbatim}
\end{tcolorbox}

    \begin{center}
    \adjustimage{max size={0.9\linewidth}{0.9\paperheight}}{EDA_files/EDA_13_0.png}
    \end{center}
    { \hspace*{\fill} \\}
    
    \begin{center}
    \adjustimage{max size={0.9\linewidth}{0.9\paperheight}}{EDA_files/EDA_13_1.png}
    \end{center}
    { \hspace*{\fill} \\}
    
    \begin{center}
    \adjustimage{max size={0.9\linewidth}{0.9\paperheight}}{EDA_files/EDA_13_2.png}
    \end{center}
    { \hspace*{\fill} \\}
    
    \begin{tcolorbox}[breakable, size=fbox, boxrule=1pt, pad at break*=1mm,colback=cellbackground, colframe=cellborder]
\prompt{In}{incolor}{69}{\boxspacing}
\begin{Verbatim}[commandchars=\\\{\}]
\PY{c+c1}{\PYZsh{} The scatter plots over the dependent variable were odd (it is a binary target)}
\PY{c+c1}{\PYZsh{} Lets use box plots to prob those relationships instead}

\PY{n}{fig}\PY{p}{,} \PY{n}{ax1} \PY{o}{=} \PY{n}{plt}\PY{o}{.}\PY{n}{subplots}\PY{p}{(}\PY{n}{figsize}\PY{o}{=}\PY{p}{(}\PY{l+m+mi}{10}\PY{p}{,} \PY{l+m+mi}{6}\PY{p}{)}\PY{p}{)}
\PY{n}{bp} \PY{o}{=} \PY{n}{ax1}\PY{o}{.}\PY{n}{boxplot}\PY{p}{(}
    \PY{p}{[}
        \PY{n}{data}\PY{o}{.}\PY{n}{Hue}\PY{p}{[}\PY{n}{data}\PY{p}{[}\PY{l+s+s2}{\PYZdq{}}\PY{l+s+s2}{isWine}\PY{l+s+s2}{\PYZdq{}}\PY{p}{]}\PY{o}{==}\PY{k+kc}{True}\PY{p}{]}\PY{p}{,} 
        \PY{n}{data}\PY{o}{.}\PY{n}{Hue}\PY{p}{[}\PY{n}{data}\PY{p}{[}\PY{l+s+s2}{\PYZdq{}}\PY{l+s+s2}{isWine}\PY{l+s+s2}{\PYZdq{}}\PY{p}{]}\PY{o}{==}\PY{k+kc}{False}\PY{p}{]}\PY{p}{,}

        \PY{n}{data}\PY{o}{.}\PY{n}{Flavanoids}\PY{p}{[}\PY{n}{data}\PY{p}{[}\PY{l+s+s2}{\PYZdq{}}\PY{l+s+s2}{isWine}\PY{l+s+s2}{\PYZdq{}}\PY{p}{]}\PY{o}{==}\PY{k+kc}{True}\PY{p}{]}\PY{p}{,} 
        \PY{n}{data}\PY{o}{.}\PY{n}{Flavanoids}\PY{p}{[}\PY{n}{data}\PY{p}{[}\PY{l+s+s2}{\PYZdq{}}\PY{l+s+s2}{isWine}\PY{l+s+s2}{\PYZdq{}}\PY{p}{]}\PY{o}{==}\PY{k+kc}{False}\PY{p}{]}\PY{p}{,}

        \PY{n}{data}\PY{o}{.}\PY{n}{Alcohol}\PY{p}{[}\PY{n}{data}\PY{p}{[}\PY{l+s+s2}{\PYZdq{}}\PY{l+s+s2}{isWine}\PY{l+s+s2}{\PYZdq{}}\PY{p}{]}\PY{o}{==}\PY{k+kc}{True}\PY{p}{]}\PY{p}{,} 
        \PY{n}{data}\PY{o}{.}\PY{n}{Alcohol}\PY{p}{[}\PY{n}{data}\PY{p}{[}\PY{l+s+s2}{\PYZdq{}}\PY{l+s+s2}{isWine}\PY{l+s+s2}{\PYZdq{}}\PY{p}{]}\PY{o}{==}\PY{k+kc}{False}\PY{p}{]}\PY{p}{,}

    \PY{p}{]}\PY{p}{,}
    \PY{n}{notch}\PY{o}{=}\PY{k+kc}{False}\PY{p}{,} \PY{n}{vert}\PY{o}{=}\PY{k+kc}{True}\PY{p}{,} \PY{n}{whis}\PY{o}{=}\PY{l+m+mf}{1.5}
\PY{p}{)}
\PY{n}{ax1\PYZus{}conf} \PY{o}{=} \PY{n}{ax1}\PY{o}{.}\PY{n}{set}\PY{p}{(}
    \PY{n}{axisbelow}\PY{o}{=}\PY{k+kc}{True}\PY{p}{,}
    \PY{n}{title}\PY{o}{=}\PY{l+s+s1}{\PYZsq{}}\PY{l+s+s1}{Distribution of several variables by target label}\PY{l+s+s1}{\PYZsq{}}\PY{p}{,}
    \PY{n}{xlabel}\PY{o}{=}\PY{l+s+s1}{\PYZsq{}}\PY{l+s+s1}{Attribute/Label}\PY{l+s+s1}{\PYZsq{}}\PY{p}{,}
    \PY{n}{ylabel}\PY{o}{=}\PY{l+s+s1}{\PYZsq{}}\PY{l+s+s1}{Value (unscaled)}\PY{l+s+s1}{\PYZsq{}}\PY{p}{,}
    \PY{n}{xticklabels}\PY{o}{=}\PY{p}{[}
        \PY{l+s+s1}{\PYZsq{}}\PY{l+s+s1}{Hue/True}\PY{l+s+s1}{\PYZsq{}}\PY{p}{,}
        \PY{l+s+s1}{\PYZsq{}}\PY{l+s+s1}{Hue/False}\PY{l+s+s1}{\PYZsq{}}\PY{p}{,}
        \PY{l+s+s1}{\PYZsq{}}\PY{l+s+s1}{Flavanoids/True}\PY{l+s+s1}{\PYZsq{}}\PY{p}{,}
        \PY{l+s+s1}{\PYZsq{}}\PY{l+s+s1}{Flavanoids/False}\PY{l+s+s1}{\PYZsq{}}\PY{p}{,}
        \PY{l+s+s1}{\PYZsq{}}\PY{l+s+s1}{Alcohol/True}\PY{l+s+s1}{\PYZsq{}}\PY{p}{,}
        \PY{l+s+s1}{\PYZsq{}}\PY{l+s+s1}{Alcohol/False}\PY{l+s+s1}{\PYZsq{}}\PY{p}{,}
    \PY{p}{]}
\PY{p}{)}
\PY{n}{utils}\PY{o}{.}\PY{n}{save\PYZus{}fig}\PY{p}{(}\PY{l+s+s1}{\PYZsq{}}\PY{l+s+s1}{box\PYZus{}plots}\PY{l+s+s1}{\PYZsq{}}\PY{p}{)}
\end{Verbatim}
\end{tcolorbox}

    \begin{center}
    \adjustimage{max size={0.9\linewidth}{0.9\paperheight}}{EDA_files/EDA_14_0.png}
    \end{center}
    { \hspace*{\fill} \\}
    
    \hypertarget{eda-engineering-features}{%
\paragraph{EDA: Engineering features}\label{eda-engineering-features}}

    \begin{tcolorbox}[breakable, size=fbox, boxrule=1pt, pad at break*=1mm,colback=cellbackground, colframe=cellborder]
\prompt{In}{incolor}{75}{\boxspacing}
\begin{Verbatim}[commandchars=\\\{\}]
\PY{c+c1}{\PYZsh{} The scatter plot for Color\PYZus{}intensity over Flavanoids seemed to have two trends present.}
\PY{c+c1}{\PYZsh{} I wonder what the distributions over their ratio looks like for each}
\PY{c+c1}{\PYZsh{} value of the target (isWine)}

\PY{n}{data}\PY{p}{[}\PY{l+s+s1}{\PYZsq{}}\PY{l+s+s1}{color\PYZus{}per\PYZus{}flavanoid}\PY{l+s+s1}{\PYZsq{}}\PY{p}{]} \PY{o}{=} \PY{n}{data}\PY{p}{[}\PY{l+s+s1}{\PYZsq{}}\PY{l+s+s1}{Color\PYZus{}intensity}\PY{l+s+s1}{\PYZsq{}}\PY{p}{]} \PY{o}{/} \PY{n}{data}\PY{p}{[}\PY{l+s+s1}{\PYZsq{}}\PY{l+s+s1}{Flavanoids}\PY{l+s+s1}{\PYZsq{}}\PY{p}{]}

\PY{c+c1}{\PYZsh{} Checking out the original histograms diplayed above we can see the distributions}
\PY{c+c1}{\PYZsh{} of Flavanoids have the same range for isWine == True and isWine == False}
\PY{c+c1}{\PYZsh{} but the variablity is quite differenent. Let\PYZsq{}s see if log(Flavanoid) gives}
\PY{c+c1}{\PYZsh{} us anything interesting.}

\PY{n}{data}\PY{p}{[}\PY{l+s+s1}{\PYZsq{}}\PY{l+s+s1}{log\PYZus{}flavanoid}\PY{l+s+s1}{\PYZsq{}}\PY{p}{]} \PY{o}{=} \PY{n}{np}\PY{o}{.}\PY{n}{log}\PY{p}{(}\PY{n}{data}\PY{p}{[}\PY{l+s+s1}{\PYZsq{}}\PY{l+s+s1}{Flavanoids}\PY{l+s+s1}{\PYZsq{}}\PY{p}{]}\PY{p}{)}


\PY{n}{fig}\PY{p}{,} \PY{n}{ax1} \PY{o}{=} \PY{n}{plt}\PY{o}{.}\PY{n}{subplots}\PY{p}{(}\PY{n}{figsize}\PY{o}{=}\PY{p}{(}\PY{l+m+mi}{10}\PY{p}{,} \PY{l+m+mi}{6}\PY{p}{)}\PY{p}{)}
\PY{n}{bp} \PY{o}{=} \PY{n}{ax1}\PY{o}{.}\PY{n}{boxplot}\PY{p}{(}
    \PY{p}{[}
        \PY{n}{data}\PY{o}{.}\PY{n}{color\PYZus{}per\PYZus{}flavanoid}\PY{p}{[}\PY{n}{data}\PY{p}{[}\PY{l+s+s2}{\PYZdq{}}\PY{l+s+s2}{isWine}\PY{l+s+s2}{\PYZdq{}}\PY{p}{]}\PY{o}{==}\PY{k+kc}{True}\PY{p}{]}\PY{p}{,} 
        \PY{n}{data}\PY{o}{.}\PY{n}{color\PYZus{}per\PYZus{}flavanoid}\PY{p}{[}\PY{n}{data}\PY{p}{[}\PY{l+s+s2}{\PYZdq{}}\PY{l+s+s2}{isWine}\PY{l+s+s2}{\PYZdq{}}\PY{p}{]}\PY{o}{==}\PY{k+kc}{False}\PY{p}{]}\PY{p}{,}

        \PY{n}{data}\PY{o}{.}\PY{n}{log\PYZus{}flavanoid}\PY{p}{[}\PY{n}{data}\PY{p}{[}\PY{l+s+s2}{\PYZdq{}}\PY{l+s+s2}{isWine}\PY{l+s+s2}{\PYZdq{}}\PY{p}{]}\PY{o}{==}\PY{k+kc}{True}\PY{p}{]}\PY{p}{,} 
        \PY{n}{data}\PY{o}{.}\PY{n}{log\PYZus{}flavanoid}\PY{p}{[}\PY{n}{data}\PY{p}{[}\PY{l+s+s2}{\PYZdq{}}\PY{l+s+s2}{isWine}\PY{l+s+s2}{\PYZdq{}}\PY{p}{]}\PY{o}{==}\PY{k+kc}{False}\PY{p}{]}\PY{p}{,}
    \PY{p}{]}\PY{p}{,}
    \PY{n}{notch}\PY{o}{=}\PY{k+kc}{False}\PY{p}{,} \PY{n}{vert}\PY{o}{=}\PY{k+kc}{True}\PY{p}{,} \PY{n}{whis}\PY{o}{=}\PY{l+m+mf}{1.5}
\PY{p}{)}
\PY{n}{ax1\PYZus{}conf} \PY{o}{=} \PY{n}{ax1}\PY{o}{.}\PY{n}{set}\PY{p}{(}
    \PY{n}{axisbelow}\PY{o}{=}\PY{k+kc}{True}\PY{p}{,}
    \PY{n}{title}\PY{o}{=}\PY{l+s+s1}{\PYZsq{}}\PY{l+s+s1}{Distribution of several variables by target label}\PY{l+s+s1}{\PYZsq{}}\PY{p}{,}
    \PY{n}{xlabel}\PY{o}{=}\PY{l+s+s1}{\PYZsq{}}\PY{l+s+s1}{Attribute/Label}\PY{l+s+s1}{\PYZsq{}}\PY{p}{,}
    \PY{n}{ylabel}\PY{o}{=}\PY{l+s+s1}{\PYZsq{}}\PY{l+s+s1}{Value (unscaled)}\PY{l+s+s1}{\PYZsq{}}\PY{p}{,}
    \PY{n}{xticklabels}\PY{o}{=}\PY{p}{[}
        \PY{l+s+s1}{\PYZsq{}}\PY{l+s+s1}{color\PYZus{}per\PYZus{}flavanoid/True}\PY{l+s+s1}{\PYZsq{}}\PY{p}{,}
        \PY{l+s+s1}{\PYZsq{}}\PY{l+s+s1}{color\PYZus{}per\PYZus{}flavanoid/False}\PY{l+s+s1}{\PYZsq{}}\PY{p}{,}
        \PY{l+s+s1}{\PYZsq{}}\PY{l+s+s1}{log\PYZus{}flavanoid/True}\PY{l+s+s1}{\PYZsq{}}\PY{p}{,}
        \PY{l+s+s1}{\PYZsq{}}\PY{l+s+s1}{log\PYZus{}flavanoid/False}\PY{l+s+s1}{\PYZsq{}}\PY{p}{,}
    \PY{p}{]}
\PY{p}{)}
\PY{n}{utils}\PY{o}{.}\PY{n}{save\PYZus{}fig}\PY{p}{(}\PY{l+s+s1}{\PYZsq{}}\PY{l+s+s1}{box\PYZus{}plots\PYZus{}new\PYZus{}features}\PY{l+s+s1}{\PYZsq{}}\PY{p}{)}
\PY{n}{plt}\PY{o}{.}\PY{n}{show}\PY{p}{(}\PY{p}{)}
\PY{n+nb}{print}\PY{p}{(}
    \PY{n}{data}\PY{p}{[}\PY{p}{[}\PY{l+s+s1}{\PYZsq{}}\PY{l+s+s1}{color\PYZus{}per\PYZus{}flavanoid}\PY{l+s+s1}{\PYZsq{}}\PY{p}{,} \PY{l+s+s1}{\PYZsq{}}\PY{l+s+s1}{log\PYZus{}flavanoid}\PY{l+s+s1}{\PYZsq{}}\PY{p}{,} \PY{l+s+s1}{\PYZsq{}}\PY{l+s+s1}{isWine}\PY{l+s+s1}{\PYZsq{}}\PY{p}{]}\PY{p}{]}\PY{o}{.}\PY{n}{corr}\PY{p}{(}\PY{p}{)}\PY{p}{[}\PY{l+s+s1}{\PYZsq{}}\PY{l+s+s1}{isWine}\PY{l+s+s1}{\PYZsq{}}\PY{p}{]}\PY{o}{.}\PY{n}{sort\PYZus{}values}\PY{p}{(}\PY{n}{ascending}\PY{o}{=}\PY{k+kc}{False}\PY{p}{)}
\PY{p}{)}
\end{Verbatim}
\end{tcolorbox}

    \begin{center}
    \adjustimage{max size={0.9\linewidth}{0.9\paperheight}}{EDA_files/EDA_16_0.png}
    \end{center}
    { \hspace*{\fill} \\}
    
    \begin{Verbatim}[commandchars=\\\{\}]
isWine                 1.000000
log\_flavanoid          0.172148
color\_per\_flavanoid   -0.467811
Name: isWine, dtype: float64
    \end{Verbatim}

    \hypertarget{ml-pipeline-data-preparation}{%
\paragraph{ML Pipeline: Data
preparation}\label{ml-pipeline-data-preparation}}

    \begin{tcolorbox}[breakable, size=fbox, boxrule=1pt, pad at break*=1mm,colback=cellbackground, colframe=cellborder]
\prompt{In}{incolor}{50}{\boxspacing}
\begin{Verbatim}[commandchars=\\\{\}]
\PY{c+c1}{\PYZsh{} Create a Test Set }

\PY{k+kn}{from} \PY{n+nn}{sklearn}\PY{n+nn}{.}\PY{n+nn}{model\PYZus{}selection} \PY{k+kn}{import} \PY{n}{train\PYZus{}test\PYZus{}split}

\PY{n}{test\PYZus{}ratio} \PY{o}{=} \PY{l+m+mf}{0.15}       \PY{c+c1}{\PYZsh{} Let\PYZsq{}s set aside 15\PYZpc{} of the records for testing}

\PY{n}{random\PYZus{}seed} \PY{o}{=} \PY{l+m+mi}{1}         \PY{c+c1}{\PYZsh{} To prevent data leakage I will set a seed while developing}
                        \PY{c+c1}{\PYZsh{} so random\PYZus{}seed = 1 will give me the same train/test split over and over}
\PY{c+c1}{\PYZsh{} random\PYZus{}seed = None      \PYZsh{} but use a fresh seed (seed=None) when its time to submit}
            
\PY{c+c1}{\PYZsh{} Set some test\PYZus{}data aside (and not look at it until very end!)}
\PY{n}{training\PYZus{}data}\PY{p}{,} \PY{n}{test\PYZus{}data} \PY{o}{=} \PY{n}{train\PYZus{}test\PYZus{}split}\PY{p}{(}\PY{n}{data}\PY{p}{,} \PY{n}{test\PYZus{}size}\PY{o}{=}\PY{n}{test\PYZus{}ratio}\PY{p}{,} \PY{n}{random\PYZus{}state}\PY{o}{=}\PY{n}{random\PYZus{}seed}\PY{p}{)}

\PY{c+c1}{\PYZsh{} Take labels off training\PYZus{}data}
\PY{n}{x\PYZus{}training} \PY{o}{=} \PY{n}{training\PYZus{}data}\PY{o}{.}\PY{n}{drop}\PY{p}{(}\PY{p}{[}\PY{l+s+s1}{\PYZsq{}}\PY{l+s+s1}{isWine}\PY{l+s+s1}{\PYZsq{}}\PY{p}{,} \PY{l+s+s1}{\PYZsq{}}\PY{l+s+s1}{binaryClass}\PY{l+s+s1}{\PYZsq{}}\PY{p}{]}\PY{p}{,} \PY{n}{axis}\PY{o}{=}\PY{l+m+mi}{1}\PY{p}{)}
\PY{n}{training\PYZus{}labels} \PY{o}{=} \PY{n}{training\PYZus{}data}\PY{p}{[}\PY{l+s+s1}{\PYZsq{}}\PY{l+s+s1}{isWine}\PY{l+s+s1}{\PYZsq{}}\PY{p}{]}\PY{o}{.}\PY{n}{copy}\PY{p}{(}\PY{p}{)}


\PY{n}{x\PYZus{}training}
\end{Verbatim}
\end{tcolorbox}

            \begin{tcolorbox}[breakable, size=fbox, boxrule=.5pt, pad at break*=1mm, opacityfill=0]
\prompt{Out}{outcolor}{50}{\boxspacing}
\begin{Verbatim}[commandchars=\\\{\}]
     Alcohol  Malic\_acid   Ash  Alcalinity\_of\_ash  Magnesium  Total\_phenols  \textbackslash{}
29     14.02        1.68  2.21               16.0       96.0           2.65
16     14.30        1.92  2.72               20.0      120.0           2.80
147    12.87        4.61  2.48               21.5       86.0           1.70
97     12.29        1.41  1.98               16.0       85.0           2.55
159    13.48        1.67  2.64               22.5       89.0           2.60
..       {\ldots}         {\ldots}   {\ldots}                {\ldots}        {\ldots}            {\ldots}
133    12.70        3.55  2.36               21.5      106.0           1.70
137    12.53        5.51  2.64               25.0       96.0           1.79
72     13.49        1.66  2.24               24.0       87.0           1.88
140    12.93        2.81  2.70               21.0       96.0           1.54
37     13.05        1.65  2.55               18.0       98.0           2.45

     Flavanoids  Nonflavanoid\_phenols  Proanthocyanins  Color\_intensity   Hue  \textbackslash{}
29         2.33                  0.26             1.98             4.70  1.04
16         3.14                  0.33             1.97             6.20  1.07
147        0.65                  0.47             0.86             7.65  0.54
97         2.50                  0.29             1.77             2.90  1.23
159        1.10                  0.52             2.29            11.75  0.57
..          {\ldots}                   {\ldots}              {\ldots}              {\ldots}   {\ldots}
133        1.20                  0.17             0.84             5.00  0.78
137        0.60                  0.63             1.10             5.00  0.82
72         1.84                  0.27             1.03             3.74  0.98
140        0.50                  0.53             0.75             4.60  0.77
37         2.43                  0.29             1.44             4.25  1.12

     OD280/OD315\_of\_diluted\_wines  Proline
29                           3.59   1035.0
16                           2.65   1280.0
147                          1.86    625.0
97                           2.74    428.0
159                          1.78    620.0
..                            {\ldots}      {\ldots}
133                          1.29    600.0
137                          1.69    515.0
72                           2.78    472.0
140                          2.31    600.0
37                           2.51   1105.0

[151 rows x 13 columns]
\end{Verbatim}
\end{tcolorbox}
        
    \begin{tcolorbox}[breakable, size=fbox, boxrule=1pt, pad at break*=1mm,colback=cellbackground, colframe=cellborder]
\prompt{In}{incolor}{58}{\boxspacing}
\begin{Verbatim}[commandchars=\\\{\}]
\PY{c+c1}{\PYZsh{} Cleaning begins with null/missing values}

\PY{n}{null\PYZus{}rows\PYZus{}idx} \PY{o}{=} \PY{n}{x\PYZus{}training}\PY{o}{.}\PY{n}{isnull}\PY{p}{(}\PY{p}{)}\PY{o}{.}\PY{n}{any}\PY{p}{(}\PY{n}{axis}\PY{o}{=}\PY{l+m+mi}{1}\PY{p}{)}

\PY{c+c1}{\PYZsh{} We confirm what we discovered above, no missing values in this dataset}
\PY{c+c1}{\PYZsh{} (For when we move to abstraction later, I will still include an imputer)}
\PY{n+nb}{print}\PY{p}{(}
    \PY{n}{null\PYZus{}rows\PYZus{}idx}\PY{p}{[}\PY{n}{null\PYZus{}rows\PYZus{}idx}\PY{o}{==}\PY{k+kc}{True}\PY{p}{]}\PY{o}{.}\PY{n}{shape}
\PY{p}{)}
\end{Verbatim}
\end{tcolorbox}

    \begin{Verbatim}[commandchars=\\\{\}]
(0,)
    \end{Verbatim}

    \begin{tcolorbox}[breakable, size=fbox, boxrule=1pt, pad at break*=1mm,colback=cellbackground, colframe=cellborder]
\prompt{In}{incolor}{61}{\boxspacing}
\begin{Verbatim}[commandchars=\\\{\}]
\PY{c+c1}{\PYZsh{} lets build the preprocessing pipeline for numerical features}

\PY{k+kn}{from} \PY{n+nn}{sklearn}\PY{n+nn}{.}\PY{n+nn}{pipeline} \PY{k+kn}{import} \PY{n}{Pipeline}\PY{p}{,} \PY{n}{make\PYZus{}pipeline}
\PY{k+kn}{from} \PY{n+nn}{sklearn}\PY{n+nn}{.}\PY{n+nn}{preprocessing} \PY{k+kn}{import} \PY{n}{StandardScaler}
\PY{k+kn}{from} \PY{n+nn}{sklearn}\PY{n+nn}{.}\PY{n+nn}{impute} \PY{k+kn}{import} \PY{n}{SimpleImputer}

\PY{n}{numeric\PYZus{}pipeline} \PY{o}{=} \PY{n}{Pipeline}\PY{p}{(}\PY{p}{[}
    \PY{p}{(}\PY{l+s+s2}{\PYZdq{}}\PY{l+s+s2}{impute}\PY{l+s+s2}{\PYZdq{}}\PY{p}{,} \PY{n}{SimpleImputer}\PY{p}{(}\PY{n}{strategy}\PY{o}{=}\PY{l+s+s2}{\PYZdq{}}\PY{l+s+s2}{median}\PY{l+s+s2}{\PYZdq{}}\PY{p}{)}\PY{p}{)}\PY{p}{,}
    \PY{p}{(}\PY{l+s+s2}{\PYZdq{}}\PY{l+s+s2}{standardize}\PY{l+s+s2}{\PYZdq{}}\PY{p}{,} \PY{n}{StandardScaler}\PY{p}{(}\PY{p}{)}\PY{p}{)}\PY{p}{,}
\PY{p}{]}\PY{p}{)}
\end{Verbatim}
\end{tcolorbox}

    \begin{tcolorbox}[breakable, size=fbox, boxrule=1pt, pad at break*=1mm,colback=cellbackground, colframe=cellborder]
\prompt{In}{incolor}{62}{\boxspacing}
\begin{Verbatim}[commandchars=\\\{\}]
\PY{c+c1}{\PYZsh{} this example use case is not using caterorical features}
\PY{c+c1}{\PYZsh{} but this is how we could construct the pipline for those:}

\PY{k+kn}{from} \PY{n+nn}{sklearn}\PY{n+nn}{.}\PY{n+nn}{impute} \PY{k+kn}{import} \PY{n}{SimpleImputer}
\PY{k+kn}{from} \PY{n+nn}{sklearn}\PY{n+nn}{.}\PY{n+nn}{preprocessing} \PY{k+kn}{import} \PY{n}{OneHotEncoder}\PY{p}{,} \PY{n}{OrdinalEncoder}

\PY{n}{categoric\PYZus{}pipeline} \PY{o}{=} \PY{n}{Pipeline}\PY{p}{(}\PY{p}{[}
    \PY{p}{(}\PY{l+s+s2}{\PYZdq{}}\PY{l+s+s2}{ordinal\PYZus{}encoder}\PY{l+s+s2}{\PYZdq{}}\PY{p}{,} \PY{n}{OrdinalEncoder}\PY{p}{(}\PY{p}{)}\PY{p}{)}\PY{p}{,}
    \PY{p}{(}\PY{l+s+s2}{\PYZdq{}}\PY{l+s+s2}{impute}\PY{l+s+s2}{\PYZdq{}}\PY{p}{,} \PY{n}{SimpleImputer}\PY{p}{(}\PY{n}{strategy}\PY{o}{=}\PY{l+s+s2}{\PYZdq{}}\PY{l+s+s2}{most\PYZus{}frequent}\PY{l+s+s2}{\PYZdq{}}\PY{p}{)}\PY{p}{)}\PY{p}{,}
    \PY{p}{(}\PY{l+s+s2}{\PYZdq{}}\PY{l+s+s2}{encode}\PY{l+s+s2}{\PYZdq{}}\PY{p}{,} \PY{n}{OneHotEncoder}\PY{p}{(}\PY{n}{handle\PYZus{}unknown}\PY{o}{=}\PY{l+s+s2}{\PYZdq{}}\PY{l+s+s2}{ignore}\PY{l+s+s2}{\PYZdq{}}\PY{p}{)}\PY{p}{)}\PY{p}{,}
\PY{p}{]}\PY{p}{)}
\end{Verbatim}
\end{tcolorbox}

    \begin{tcolorbox}[breakable, size=fbox, boxrule=1pt, pad at break*=1mm,colback=cellbackground, colframe=cellborder]
\prompt{In}{incolor}{63}{\boxspacing}
\begin{Verbatim}[commandchars=\\\{\}]
\PY{c+c1}{\PYZsh{} A ColumnTransformer can make a single pipeline}

\PY{k+kn}{from} \PY{n+nn}{sklearn}\PY{n+nn}{.}\PY{n+nn}{compose} \PY{k+kn}{import} \PY{n}{make\PYZus{}column\PYZus{}selector}\PY{p}{,} \PY{n}{make\PYZus{}column\PYZus{}transformer}

\PY{n}{preprocessing} \PY{o}{=} \PY{n}{make\PYZus{}column\PYZus{}transformer}\PY{p}{(}
    \PY{p}{(}\PY{n}{numeric\PYZus{}pipeline}\PY{p}{,} \PY{n}{make\PYZus{}column\PYZus{}selector}\PY{p}{(}\PY{n}{dtype\PYZus{}include}\PY{o}{=}\PY{n}{np}\PY{o}{.}\PY{n}{number}\PY{p}{)}\PY{p}{)}\PY{p}{,}
    \PY{p}{(}\PY{n}{categoric\PYZus{}pipeline}\PY{p}{,} \PY{n}{make\PYZus{}column\PYZus{}selector}\PY{p}{(}\PY{n}{dtype\PYZus{}include}\PY{o}{=}\PY{n+nb}{object}\PY{p}{)}\PY{p}{)}\PY{p}{,}
\PY{p}{)}

\PY{n}{preprocessing}
\end{Verbatim}
\end{tcolorbox}

            \begin{tcolorbox}[breakable, size=fbox, boxrule=.5pt, pad at break*=1mm, opacityfill=0]
\prompt{Out}{outcolor}{63}{\boxspacing}
\begin{Verbatim}[commandchars=\\\{\}]
ColumnTransformer(transformers=[('pipeline-1',
                                 Pipeline(steps=[('impute',
SimpleImputer(strategy='median')),
                                                 ('standardize',
                                                  StandardScaler())]),
<sklearn.compose.\_column\_transformer.make\_column\_selector object at
0x7ff32accb450>),
                                ('pipeline-2',
                                 Pipeline(steps=[('ordinal\_encoder',
                                                  OrdinalEncoder()),
                                                 ('impute',
SimpleImputer(strategy='most\_frequent')),
                                                 ('encode',
OneHotEncoder(handle\_unknown='ignore'))]),
<sklearn.compose.\_column\_transformer.make\_column\_selector object at
0x7ff32accb210>)])
\end{Verbatim}
\end{tcolorbox}
        
    \begin{tcolorbox}[breakable, size=fbox, boxrule=1pt, pad at break*=1mm,colback=cellbackground, colframe=cellborder]
\prompt{In}{incolor}{64}{\boxspacing}
\begin{Verbatim}[commandchars=\\\{\}]
\PY{c+c1}{\PYZsh{} We can test our preprocessing pipeline on our training data.}

\PY{n}{training\PYZus{}data\PYZus{}cleaned} \PY{o}{=} \PY{n}{preprocessing}\PY{o}{.}\PY{n}{fit\PYZus{}transform}\PY{p}{(}\PY{n}{x\PYZus{}training}\PY{p}{)}

\PY{n}{training\PYZus{}data\PYZus{}cleaned\PYZus{}df} \PY{o}{=} \PY{n}{pd}\PY{o}{.}\PY{n}{DataFrame}\PY{p}{(}
    \PY{n}{training\PYZus{}data\PYZus{}cleaned}\PY{p}{,}
    \PY{n}{columns}\PY{o}{=}\PY{n}{preprocessing}\PY{o}{.}\PY{n}{get\PYZus{}feature\PYZus{}names\PYZus{}out}\PY{p}{(}\PY{p}{)}\PY{p}{,}
    \PY{n}{index}\PY{o}{=}\PY{n}{training\PYZus{}data}\PY{o}{.}\PY{n}{index}
\PY{p}{)}

\PY{n}{training\PYZus{}data\PYZus{}cleaned\PYZus{}df}\PY{o}{.}\PY{n}{head}\PY{p}{(}\PY{p}{)}
\end{Verbatim}
\end{tcolorbox}

            \begin{tcolorbox}[breakable, size=fbox, boxrule=.5pt, pad at break*=1mm, opacityfill=0]
\prompt{Out}{outcolor}{64}{\boxspacing}
\begin{Verbatim}[commandchars=\\\{\}]
     pipeline-1\_\_Alcohol  pipeline-1\_\_Malic\_acid  pipeline-1\_\_Ash  \textbackslash{}
29              1.259587               -0.619230        -0.535083
16              1.603015               -0.410962         1.316905
147            -0.150920                1.923377         0.445381
97             -0.862306               -0.853531        -1.370293
159             0.597262               -0.627908         1.026397

     pipeline-1\_\_Alcalinity\_of\_ash  pipeline-1\_\_Magnesium  \textbackslash{}
29                       -1.063608              -0.207330
16                        0.151484               1.535977
147                       0.607144              -0.933709
97                       -1.063608              -1.006346
159                       0.910916              -0.715795

     pipeline-1\_\_Total\_phenols  pipeline-1\_\_Flavanoids  \textbackslash{}
29                    0.589146                0.337732
16                    0.827519                1.147851
147                  -0.920554               -1.342516
97                    0.430230                0.507757
159                   0.509688               -0.892449

     pipeline-1\_\_Nonflavanoid\_phenols  pipeline-1\_\_Proanthocyanins  \textbackslash{}
29                          -0.874546                     0.710735
16                          -0.286882                     0.693358
147                          0.888445                    -1.235501
97                          -0.622690                     0.345816
159                          1.308205                     1.249426

     pipeline-1\_\_Color\_intensity  pipeline-1\_\_Hue  \textbackslash{}
29                     -0.165290         0.365559
16                      0.455370         0.494389
147                     1.055342        -1.781607
97                     -0.910082         1.181482
159                     2.751813        -1.652777

     pipeline-1\_\_OD280/OD315\_of\_diluted\_wines  pipeline-1\_\_Proline
29                                   1.405185             0.963100
16                                   0.085971             1.769644
147                                 -1.022730            -0.386627
97                                   0.212279            -1.035154
159                                 -1.135004            -0.403087
\end{Verbatim}
\end{tcolorbox}
        
    \hypertarget{deployment-model-selection-without-data-transformations}{%
\paragraph{Deployment: Model selection (without data
transformations)}\label{deployment-model-selection-without-data-transformations}}

    \begin{tcolorbox}[breakable, size=fbox, boxrule=1pt, pad at break*=1mm,colback=cellbackground, colframe=cellborder]
\prompt{In}{incolor}{ }{\boxspacing}
\begin{Verbatim}[commandchars=\\\{\}]
\PY{c+c1}{\PYZsh{} This is where next work will resume...}
\end{Verbatim}
\end{tcolorbox}


    % Add a bibliography block to the postdoc
    
    
    
\end{document}
